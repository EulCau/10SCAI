\documentclass[aspectratio=169]{beamer}
\usetheme{Madrid}
\usepackage{ctex, amsmath, amsfonts, amssymb}
\usepackage{graphicx}

\def\dif{\mathinner{\mathrm{d}}\hphantom{\mskip-\thinmuskip}}

\title{高维 PDE 求解算法:\\从非线性蒙特卡罗到机器学习}
\author{Weinan E \and Jiequn Han \and Arnulf Jentzen}
\date{2020年9月14日}

\begin{document}

	\begin{frame}
		\titlepage
	\end{frame}

	\begin{frame}{大纲}
		\tableofcontents
	\end{frame}

	\section{引言与背景}
	\begin{frame}{引言与背景}
		\begin{itemize}
			\item 高维偏微分方程 (PDEs) 在控制理论, 金融工程, 量子力学等领域有重要应用
			\item 传统数值方法 (如有限差分、有限元) 在高维情况下受``维数灾难''限制
			\item 本论文提出利用非线性蒙特卡罗方法和深度学习技术求解高维 PDE, 从而突破维数灾难
		\end{itemize}
	\end{frame}

	\section{主要方法与技术}
	\begin{frame}{深度 BSDE 方法 (Deep BSDE)}
		\begin{itemize}
			\item 将非线性抛物型 PDE
				\begin{equation*}
					\frac{\partial u}{\partial t} + \frac{1}{2} \mathrm{Tr}\left(\sigma\sigma^{\top}\,\mathrm{Hess}_x u\right) + \langle\nabla u, \mu\rangle + f\left(t, x, u, \sigma^{\top} \nabla u\right)=0, \quad u\left(T,x\right)=g\left(x\right)
				\end{equation*}
				与后向随机微分方程 (BSDE) 联系起来
			\item 通过 It\^{o} 引理可得:
				\begin{equation*}
					\begin{aligned}
						u\left(t, X_{t}\right) - u\left(0, X_{0}\right) = &-\int_{0}^{t} f\left(s, X_{s}, u\left(s, X_s\right), \sigma^{\top}\nabla u\left(s, X_{s}\right)\right) \dif s \\
						&+ \int_{0}^{t} \left(\nabla u\left(s, X_{s}\right)\right)^{\top}\sigma(s, X_{s}) \dif W_{s}.
					\end{aligned}
				\end{equation*}
		\end{itemize}
	\end{frame}

	\begin{frame}{深度 BSDE 方法 (Deep BSDE)}
			\begin{itemize}
				\item Pardoux 和 Peng 提出如果令 $Y_{t} = u\left(t, X_{t}\right), Z_{t} = \left[\sigma\left(t, X_{t}\right)\right]^{\top} \left(\nabla_{x} u \right)\left(t, X_{t}\right)$, 则随机过程 $\left(X_{t}, Y_{t}, Z_{t}\right) \in \mathbb{R}^{d} \times \mathbb{R} \times \mathbb{R}^d, t \in \left[0, T\right]$, 满足下面的 BSDE:
					\begin{equation*}
						\left\{\begin{array}{l}
							X_{t} = \xi + \int_{0}^{t} \mu\left(s, X_{s}\right) \dif s + \int_{0}^{t} \sigma\left(s, X_{s}\right) \dif W_{s} \\
							Y_{t} = g\left(X_{T}\right) + \int_{t}^{T} f\left(s, X_{s}, Y_{s}, Z_{s}\right) \dif s - \int_{t}^{T} \left(Z_{s}\right)^{\top} \dif W_{s}
						\end{array}\right.
					\end{equation*}
				\item 利用深度神经网络逼近未知函数: 例如用网络 $\psi_{0}$ 表示 $u(0, X_0)$, 用子网络 $\phi_n$ 逼近 $Z_t\ (= \sigma^T\nabla u)$
				\item 采用时间离散化 (Euler 方法) 构建网络, 末端误差作为损失函数进行训练
			\end{itemize}
	\end{frame}

	\begin{frame}{多层皮卡德方法 (MLP)}
		\begin{itemize}
			\item 针对半线性热方程
				\begin{equation*}
				\frac{\partial u}{\partial t}(t,x)=\Delta u(t,x)+f(u(t,x))
				\end{equation*}
			\item 将 PDE 写成不动点形式 \(u = \Phi(u)\),并构造 Picard 迭代
				\begin{equation*}
				u_n = \Phi(0) + \sum_{l=1}^{n-1} \bigl[\Phi(u_l)-\Phi(u_{l-1})\bigr].
				\end{equation*}
			\item 利用多层次蒙特卡罗采样逼近期望值,每层使用不同精度的蒙特卡罗估计
			\item 理论证明:在一定条件下,计算复杂度仅呈多项式增长,克服了维数灾难问题
		\end{itemize}
	\end{frame}

	\begin{frame}{传统方法与神经网络结合}
		\begin{itemize}
			\item 利用 Ritz、Galerkin 以及最小二乘方法构造数值方案
			\item 例如,深度 Ritz 方法将问题转化为求解变分问题:
				\begin{equation*}
				\min_{u\in H} I(u),\quad I(u)=\int_\Omega \Bigl(\frac{1}{2}|\nabla u(x)|^2 - f(x)u(x)\Bigr)dx.
				\end{equation*}
			\item 通过神经网络表示试探函数,实现无网格(mesh-free)的近似,具有自然的自适应性
		\end{itemize}
	\end{frame}

	\section{数学公式解析}
	\begin{frame}{关键数学公式解析}
		\begin{itemize}
			\item \textbf{Hamilton-Jacobi-Bellman (HJB) 方程:}
				\begin{equation*}
				\frac{\partial u}{\partial t}+H(x,\nabla u)=0.
				\end{equation*}
				描述最优控制问题中价值函数的动态规划原理
			\item \textbf{Black-Scholes 方程:}
				\begin{equation*}
				\frac{\partial u}{\partial t}+\frac{1}{2}\sigma^2 \sum_{i=1}^{d}x_i^2 \frac{\partial^2 u}{\partial x_i^2}+r\langle \nabla u,x\rangle-ru+f(u)=0.
				\end{equation*}
				用于金融衍生品定价,考虑违约风险、交易成本等非线性效应
			\item \textbf{蒙特卡罗积分:}
				\begin{equation*}
				I(g)=\int_{[0,1]^d}g(x)dx,\quad I_n(g)=\frac{1}{n}\sum_{j=1}^n g(x_j),
				\end{equation*}
				误差分析:
				\begin{equation*}
				\mathbb{E}\Bigl[|I(g)-I_n(g)|^2\Bigr]=\frac{\mathrm{Var}(g)}{n}.
				\end{equation*}
			\item \textbf{神经网络近似表达式:}
				\begin{equation*}
				f_m(x,\theta)=\frac{1}{m}\sum_{j=1}^{m}a_j\,\sigma\bigl(\langle w_j,x\rangle\bigr),
				\end{equation*}
				以及 ResNet 形式:
				\begin{equation*}
				z_{l+1}=z_l+\frac{1}{L}\sum_{j=1}^{M}a_{j,l}\,\sigma\bigl(\langle z_l,w_{j,l}\rangle\bigr).
				\end{equation*}
		\end{itemize}
	\end{frame}

	\section{结论与未来展望}
	\begin{frame}{结论与未来展望}
		\begin{itemize}
			\item 利用非线性蒙特卡罗和深度学习方法,设计出不受维数灾难影响的高效数值算法
			\item 理论与数值实验均证明:对于控制、金融、量子等领域的高维问题,这些方法具有显著优势
			\item 未来工作:进一步完善理论证明、改进算法效率以及扩展到更广泛的应用场景
		\end{itemize}
	\end{frame}

	\begin{frame}{参考文献}
		\begin{itemize}
			\item Weinan E, Jiequn Han, Arnulf Jentzen. \emph{Algorithms for Solving High Dimensional PDEs: From Nonlinear Monte Carlo to Machine Learning}, 2020.
			\item 其他相关文献……
		\end{itemize}
	\end{frame}

	\begin{frame}
		\centering \Large 谢谢大家!
	\end{frame}

\end{document}
